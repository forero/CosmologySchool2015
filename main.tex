\documentclass[12pt]{article}
%\usepackage[spanish]{babel}
\usepackage{geometry}

\title{Proposal for a Sanford Chair 2015 in Cosmology}
\author{Jaime E. Forero-Romero\\{\small Assistant Professor, Universidad de los Andes, Bogot\'a, Colombia}}
\date{\today}


\begin{document}


\maketitle

\begin{abstract}
This document summarizes the proposal to invite Prof. Yehuda Hoffman
and Dr. Stefan Gottloeber for a visit during June 2015 sponsored by the
Sanford Chair funds. The main results of this visit will be three: 
(i) offer a one month summer course on cosmology; (ii) perform large numerical
simulations that will make use of the new HPC facilities at Uniandes
(iii) submit a scientific publication led by a Masters student at
Uniandes.  
\end{abstract}


\section{Invited Researchers}


\noindent
{\bf Prof. Yehuda Hoffman} (Tel-Aviv 1951) Professor at the Racah
Institute of Physics in the Hebrew University of Jerusalem
(Israel). PhD in 1983 from Tel-Aviv Univeristy, with 150+ publications
and H-index of 37. His research in the last 30 years has focused on
theoretical aspects of the large scale structure in the Universe. \\

\noindent
{\bf Dr. Stefan Gottl\"ober} (Berlin 1951) Staff Scientist at the
Leibniz-Institute for Astrophysics in Potsdam (Germany). PhD in 1980
at the Academy of Sciences in Berlin, with 160+ publications and
H-index of 40. His research in the last 20 years has focused on using
supercomputing techniques  to simulate the matter distribution of the
Universe on its largest scales. 

\section{Activities during the Sanford Chair period}

The visit of Prof. Hoffman and Dr. Gottl\"ber will last one
month, starting on June 1st 2015 and finishing ond June 30th 2015. 

\subsection{Research}

The research activities will focus on two areas: the 
numerical modeling of the Local Universe on cosmological scales and
the automatized description of the filamentary large scale structure of the
Universe. This are main lines of my research as presented in my FAPA
(Fondo de Asistencia a Profesores Asistentes) project. 

Our joint collaboration on these subjects started in 2009 and has
been presented in 5 refereed publications since then. 

For this visit we have two specific objectives

\begin{itemize}
\item {\bf Launch large numerical simulations} of constrained realization
of the Local Universe. This will make use of the new HPC facility at Uniandes.
\item {\bf Finish a publication} on the nature of the associations of Dwarf
Associations in the Local Volume. This will involve Juan Nicolas
Garavito, a Physics masters student at Uniandes.
\end{itemize} 

\subsection{Teaching}


Prof. Hoffman and Dr. Gottl\"ober will offer a summer course on
the theoretical, computational and observational aspects of cosmology
as studied with large scale structures in the Universe. This course
will have {\bf 60 teaching blocks of 45 minutes each} so it can be
offered as a 3 credit course to students in the Physics Department. 

Prof. Hoffman will teach 15 blocks on theoretical and observational aspects
of physical cosmology and Dr. Gottl\"ober will teach 15 blocks on
different aspects of computational cosmology. The remaining 30
blocks will be led by Prof. Forero-Romero as hands-on sessions. This
last part will be offered as small research projects that should be
tackled during the last two weeks of the Summer Course; Prof. Hoffman
and Dr. Gottl\"ober will also be available to assist the students. 

We will open this course to physics students in the whole country,
with a limit of 25 participants, giving priority to Uniandes
students. No fee will be charged. Only five travel fellowships will be
offered to students from outside Bogot\'a. 

We expect to follow the following schedule

\begin{itemize}
\item{First week: 15 blocks on theoretical aspects of physical
  cosmology by Prof. Hoffman}
\item{Second week: 15 blocks on computational cosmology by
  Dr. Gottl\"ober.}
\item{Third week: 15 blocks as hands-on practice for the research
  projects. Focus on how to use the HPC facilities at Uniandes. Led by
  Prof. Forero-Romero.} 
\item{Fourth week: 15 blocks as hands-on practice for the research
  projects. Led by Prof. Forero-Romero.}
\end{itemize}

\section{Budget}



\begin{tabular}{lll}\hline
Description & Item Cost & Total Cost \\\hline
Hotel for Prof. Hoffman & 80USD/night & 2400 USD\\
Hotel for Dr. Gottl\"ober & 80USD/night & 2400 USD\\
Per diem for Prof. Hoffman & 50USD  & 1500 USD\\
Per diem for Dr. Gottl\"ober & 50USD & 1500 USD \\
Airfare from TelAviv (Prof. Hoffman) & 1400 USD & 1500 USD\\
Airfare from Berlin (Dr. Gottlober) & 1400 USD & 1500 USD\\
Travel grants for students & 200 USD & 1000 USD\\\hline
 & {\bf Total} & 11800 USD\\\hline
\end{tabular}

\vspace{1cm}

The Sanford Chair fund provides 8000USD that cover the hotel and
per diem expenses for Prof. Hoffman and Dr. Gottlober. The remaining
4000 USD for travel expenses will be covered from the FAPA grant of
Prof. Forero-Romero. 

We will look for external funding possibilities (DAAD, ICETEX,
COLCIENCIAS) to cover the travel costs and offer financial support to
the students attending the school. 

\newpage
\section*{Appendix A: Yehuda Hoffman CV} 

\newpage
\section*{Appendix B: Yehuda Hoffman Recent Publication List} 

\newpage
\section*{Appendix C: Stefan Gottl\"ober CV} 

\newpage
\section*{Appendix D: Stefan Gottl\"ober Recent Publication List} 

\end{document}
